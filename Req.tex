%% LyX 2.2.1 created this file.  For more info, see http://www.lyx.org/.
%% Do not edit unless you really know what you are doing.
\documentclass[english]{article}
\usepackage[T1]{fontenc}
\usepackage[latin9]{inputenc}
\usepackage{textcomp}

\makeatletter
%%%%%%%%%%%%%%%%%%%%%%%%%%%%%% User specified LaTeX commands.
\usepackage{babel}

\makeatother

\usepackage{babel}
\begin{document}

\title{5 The problem for assignments 1, 2, 4, and 5 \textendash{} PowerEnJoy }

\maketitle
You are to develop a digital management system for a car sharing service
that exclusively employs electric cars. First, the system should provide
the functionality normally provided by car\textendash sharing services.
These include:
\begin{itemize}
\item { Users must be able to register to the system by providing their
credentials, payment information and required documents. The credential
must be unique among all previously registered users. Documents and
payment information must be verified in order to successfully create
an account. Users receive back a password that can be used to access
the system. }
\item Required documents include driving license (at least 1 year of experience).
Payment information consist in credit card 
\item { Registered users must be able to find the locations of available
cars within a certain distance from their current location or from
a specified address. The system also provide information about the
estimated car autonomy in kilometers.}
\item { Among the available cars in a certain geographical region, users
must be able to reserve a single car for up to one hour before they
pick it up, by unlocking the car.}
\item If a car is not picked\textendash up within one hour from the reservation,
the system tags the car as available again, and the reservation expires;
the user pays a fee of 1 EUR. 
\item An user is able to cancel the reservation before it's expiration,
paying a cost proportional to the reservation time (the minimum amount
is relative to a quarter of an hour).
\item { A user that reaches it's reserved car must be able to tell the
system he's nearby, so the system unlocks the car and the user may
enter, checking that the user is within a certain threshold from the
car. }
\item As soon as the user unlocks the car, the system starts charging the
user for a given amount of money per minute; the user is notified
of the current fare through a screen on the car.
\item The system debits the user on his card every 10�. Whether the user
is not able to pay, the system stops the car respectfully of user's
safety, after notifying the user on the screen.
\item The system stops charging the user as soon as the car is parked in
a safe area and the user exits the car; at this point the system locks
the car automatically. Then the system applies the discount to the
total fare and debit/credit the difference relative to the payed amount.
\item { The set of safe areas for parking cars is pre\textendash defined
by the management system.}
\end{itemize}
In addition to the functionality above, the system should incentivize
the virtuous behaviors of the users. Specifically: 
\begin{enumerate}
\item { If the system detects the user took at least two other passengers
onto the car, the system applies a discount of 10\% on the last ride.}
\item { If a car is left with more than 50\% of the battery's charge, the
system applies a discount of 20\% on the last ride. }
\item { If a car is left at special safe areas where they can be recharged
and the user takes care of plugging the car into the power grid, the
system applies a discount of 30\% on the last ride.}
\item { If a car is left at more than 3 KM from the nearest power grid
station or with less than 20\% of the battery's charge, the system
charges 30\% more on the last ride to compensate for the cost required
to re\textendash charge the car on\textendash site.}
\item { If the user enables the money saving option, he/she can input his/her
final destination and the system provides information about where
to leave the car to get the discount described above at point (3).
This location is determined to ensure a uniform distribution of cars
in the city and depends both on the destination of the user and on
the availability of power plugs at that safe area.}
\end{enumerate}

\end{document}
